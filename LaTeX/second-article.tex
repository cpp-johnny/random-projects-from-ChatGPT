\documentclass{article}
\begin{document}
\textbf{The epidemics of the early 21st century revealed a world unprepared, even as the risks continue to multiply. Much worse is coming.} \\

\textbf{By Ed Yong}\\

At 6 o’clock in the morning, shortly after the sun spills over the horizon, the city of Kikwit doesn’t so much wake up as ignite. Loud music blares from car radios. Shops fly open along the main street. Dust-sprayed jeeps and motorcycles zoom eastward toward the town’s bustling markets or westward toward Kinshasa, the Democratic Republic of the Congo’s capital city. The air starts to heat up, its molecules vibrating with absorbed energy. So, too, the city. \\

By late morning, I am away from the bustle, on a quiet, exposed hilltop some five miles down a pothole-ridden road. As I walk, desiccated shrubs crunch underfoot and butterflies flit past. The only shade is cast by two lines of trees, which mark the edges of a site where more than 200 people are buried, their bodies piled into three mass graves, each about 15 feet wide and 70 feet long. Nearby, a large blue sign says \textit{in memory of the victims of the ebola epidemic in may 1995}. The sign is partly obscured by overgrown grass, just as the memory itself has been occluded by time. The ordeal that Kikwit suffered has been crowded out by the continual eruption of deadly diseases elsewhere in the Congo, and around the globe. \\

article from: https://www.theatlantic.com/magazine/archive/2018/07/when-the-next-plague-hits/561734/

\end{document}
